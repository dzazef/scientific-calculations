\documentclass[12pt]{article}
\usepackage[utf8]{inputenc}
\usepackage{polski}
\usepackage{listing}
\usepackage{amsmath}
\usepackage{multicol}
\usepackage{graphicx}
\usepackage{enumitem}
\graphicspath{ {./images/} }
\title{
	Obliczenia naukowe \\
	Sprawozdanie 3
}
\addtolength{\oddsidemargin}{-.8in}
\addtolength{\evensidemargin}{-.8in}
\addtolength{\textwidth}{1.75in}
\addtolength{\topmargin}{-.7in}
\addtolength{\textheight}{1.75in}

\date{23 listopada 2019}
\author{Józef Piechaczek}

\begin{document}

\pagenumbering{gobble}
\maketitle
\newpage
\pagenumbering{arabic}

\setlength{\abovedisplayskip}{5pt}
\setlength{\belowdisplayskip}{5pt}

\section*{Notatka do zadań 1, 2 i 3}
Do zadań 1, 2 i 3 zostały napisane funkcje testujące znajdujące się w pliku \texttt{tests.jl}. Testy sprawdzają wartości miejsc zerowych policzone za pomocą danych metod dla funkcji $f_1(x) = x$, $f_2(x)=x^2$, $f_3(x)=x+1$, $f_4(x)=sin x$.

\section{Zadanie 1}
Zadanie 1 polega na utworzeniu funkcji rozwiązującej równianie $f(x) = 0$ za pomocą metody bisekcji. \textbf{Definicja funkcji:}
\begin{verbatim}
function mbisekcji(f, a::Float64, b::Float64, delta::Float64, 
epsilon::Float64)
\end{verbatim}
\textbf{Dane:}
\begin{itemize}[leftmargin=4.0cm,labelsep=0.4cm]
\item[$f$] funkcja $f(x)$ zadana jako anonimowa funkcja
\item[$a, b$] końce przedziału początkowego
\item[$delta, epsilon$] dokładności obliczeń
\end{itemize}
\textbf{Wyniki:} 
\begin{itemize}[leftmargin=4.0cm,labelsep=0.4cm]
\item[$(r, v, it, err)$] czwórka, gdzie
\item[$r$] przybliżenie pierwiastka równiania 
\item[$v$] wartość $f(r)$
\item[$it$] liczba iteracji
\item[$err$] sygnalizacja błędu\\
0 - brak błędu\\
1 - funkcja nie zmienia znaku w przedziale $[a,b]$
\end{itemize}

\noindent \textbf{Opis metody:}\\
Metoda bisekcji korzysta z twierdzenia Darboux do szacowania wartości pierwiastka funkcji. Twierdzenie to mówi, że funkcja ciągła na danym przedziale $[a, b]$ i zmieniająca znak na tym przedziale (co opisuje nierówność $f(a)f(b) < 0$) posiada miejsce zerowe w $[a,b]$. Metoda w pojedynczej iteracji dzieli podany przedział na połowę, obliczając środek przedziału $c=\frac{1}{2}(b+a)$. Następnie obliczamy wartość funkcji w punkcie c, tj. $f(c)$. Sprawdzamy znak wartości $f(c)$ i rozpatrujemy nowy przedział $(a, c)$ albo $(c, b)$ taki, że funkcja zmienia znak w danym przedziale. Metoda kończy działanie w momencie gdy osiągamy zadaną dokładność, tj. przedział jest dostatecznie niewielki lub wartość w punkcie c jest dostatecznie bliska zeru. Metoda ma zbieżność liniową.
\\
\\
\noindent \textbf{Opis algorytmu:}\\
Aby algorytm był jak najlepszy z numerycznego punktu widzenia wprowadziłem następujące zmiany:
\begin{enumerate}
	\item Środek przedziału liczymy za pomocą wzoru $e=(b-a)/2, c=a+e$. Jest to dokładniejszy sposób niż obliczanie środka klasyczną metodą $c=(a+b)/2$, gdyż klasyczna metoda mogła powodować, iż środek przedziału znajdował się poza przedziałem dla niektórych przypadków.
	\item Wartości funkcji na krańcach przedziału porównujemy za pomocą \texttt{sign(f(c)) != sign(f(a))}, gdyż sprawdzanie tego za pomocą warunku $f(a)f(b) < 0$ mogło powodować overflow lub underflow.
\end{enumerate}
Algorytm kończy pracę na jeden z dwóch sposobów
\begin{enumerate}
\item Gdy zadana dokładność została osiągnięta
\item Gdy funkcja nie zmienia znaku w zadanym przedziale - error code 1
\end{enumerate}





\section{Zadanie 2}
Zadanie 2 polega na utworzeniu funkcji rozwiązującej równianie $f(x) = 0$ za pomocą metody Newtona. \textbf{Definicja funkcji:}
\begin{verbatim}
function mstyczych(f, pf, x0::Float64, delta::Float64, 
epsilon::Float64, maxit::Int)
\end{verbatim}
\textbf{Dane:}
\begin{itemize}[leftmargin=4.0cm,labelsep=0.4cm]
\item[$f, pf$] funkcja $f(x)$ oraz pochodna $f'(x)$ zadane jako anonimowe funkcje
\item[$x0$] przybliżenie początkowe
\item[$delta, epsilon$] dokładności obliczeń
\item[$maxit$] maksymalna liczba iteracji
\end{itemize}
\textbf{Wyniki:} 
\begin{itemize}[leftmargin=4.0cm,labelsep=0.4cm]
\item[$(r, v, it, err)$] czwórka, gdzie
\item[$r$] przybliżenie pierwiastka równiania 
\item[$v$] wartość $f(r)$
\item[$it$] liczba iteracji
\item[$err$] sygnalizacja błędu\\
0 - brak błędu\\
1 - nie osiągnieto wymaganej dokładności w $maxit$ operacji\\
2 - pochodna bliska zeru
\end{itemize}

\noindent \textbf{Opis metody:}\\
Metoda Newtona polega na lokalnej aproksymacji liniowej za pomocą stycznych. W tej metodzie iteracyjnej $x_{n+1}$ jest określone jako odcięta punktu przecięcia z osią x stycznej do krzywej $y=f(x)$ w punkcie $(x_n,f(x_n))$. Zatem kolejne przybliżenia pierwiastka równiania można określić za pomocą wzoru:
\begin{equation*}
x_{n+1}=x_n-\frac{f(x_n)}{f'(x_n)}
\end{equation*}
Metoda Newtona ma zbieżność kwadratową, zatem zbiega szybciej niż metoda bisekcji i siecznych. Wadą tej metody natomiast jest konieczność liczenia pochodnej funkcji. Inna wadą jest fakt, iż zbieżność nie zachodzi zawsze. W wielu przypadkach metoda jest rozbieżna, kiedy punkt startowy znajduje się zbyt daleko od poszukiwanego miejsca zerowego.
\\
\\
\noindent \textbf{Opis algorytmu:}\\
Program może kończyć działanie na jeden z czterech sposobów:
\begin{enumerate}
	\item Gdy spełniona jest wymagana dokładność na początku działania programu, tj. $f(x_0) < epsilon$
	\item Gdy pochodna jest bliska zeru ($|f'(x_0)|<epsilon$) - funkcja zwraca error code 2
	\item Gdy w danej iteracji spełniona jest zadana dokładność ($|x_{n+1}-x_n|<delta$ lub $|f(x_{n+1})|<epsilon$)
	\item Gdy nie osiągnięto zadanej dokładności w $maxit$ iteracji - error code 1
\end{enumerate}





\section{Zadanie 3}
Zadanie 3 polega na utworzeniu funkcji rozwiązującej równianie $f(x) = 0$ za pomocą metody siecznych. \textbf{Definicja funkcji:}
\begin{verbatim}
function msiecznych(f, x0::Float64, x1::Float64, delta::Float64, 
epsilon::Float64, maxit::Int)
\end{verbatim}
\textbf{Dane:}
\begin{itemize}[leftmargin=4.0cm,labelsep=0.4cm]
\item[$f$] funkcja $f(x)$ zadana jako anonimowa funkcja
\item[$x0, x1$] przybliżenia początkowe
\item[$delta, epsilon$] dokładności obliczeń
\item[$maxit$] maksymalna liczba iteracji
\end{itemize}
\textbf{Wyniki:} 
\begin{itemize}[leftmargin=4.0cm,labelsep=0.4cm, itemsep=0.00cm]
\item[$(r, v, it, err)$] czwórka, gdzie
\item[$r$] przybliżenie pierwiastka równiania 
\item[$v$] wartość $f(r)$
\item[$it$] liczba iteracji
\item[$err$] sygnalizacja błędu\\
0 - brak błędu\\
1 - nie osiągnieto wymaganej dokładności w $maxit$ operacji\\
\end{itemize}

\noindent \textbf{Opis metody:}\\
Metoda siecznej jest podobna do metody Newtona, jednak w tej metodzie pochodną zastępujemy ilorazem różnicowym postaci:
\begin{equation*}
f'(x_n) \approx \frac{f(x_n)-f(x_{n-1})}{x_n-x_{n-1}}
\end{equation*}
Otrzymujemy więc następujący wzór rekurencyjny
\begin{equation*}
x_{n+1} = x_n - f(x_n) \frac{x_n-x_{n-1}}{f(x_n)-f(x_{n-1})}, n \geq 1
\end{equation*}
Metoda zatem potrzebuje dwóch punktów startowych. Metoda ta ma zbieżność wynoszącą $\frac{1+\sqrt{5}}{2}$, zatem jest wolniejsza od metody Newtona, ale za to nie wymaga liczenia pochodnej funkcji.
\\
\\
\noindent \textbf{Opis algorytmu:}\\
Algorytm kończy działanie na jeden z dwóch sposobów:
\begin{enumerate}
\item Gdy spełniona jest zadana dokładność
\item Gdy zadana dokładność nie została osiągnięta w $maxit$ iteracji - error code 1
\end{enumerate}

\section{Zadanie 4}
Celem zadania 4 jest wyznaczenie pierwiastków równania $sin x - (\frac{1}{2}x)^2 = 0$ za pomocą zaprogramowanych wcześniej metod:
\begin{enumerate}
	\item bisekcji z przedziałem początkowym $[1.5, 2]$ i $\delta=\frac{1}{2}10^{-5}$, $\epsilon=\frac{1}{2}10^{-5}$ 
	\item Newtona z przybliżeniem początkowym $x_0 = 1.5$ i $\delta=\frac{1}{2}10^{-5}$, $\epsilon=\frac{1}{2}10^{-5}$ 
	\item siecznych z przybliżeniami początkowymi $x_0=1$, $x_1=2$ i $\delta=\frac{1}{2}10^{-5}$, $\epsilon=\frac{1}{2}10^{-5}$ 
\end{enumerate}

\begin{figure}[!h]
	\centering
  \includegraphics[width=0.6\linewidth]{zad4_plot.png}
  \caption{Wykres w pyplot na przedziale $[1, 2]$}\label{fig:figure1}
\end{figure}

\textbf{Wyniki:}\\
\begin{table}[h!]
	\centering
    \label{tab:table1}
    \begin{tabular}{c|c|c|c|c}
    		Metoda & Miejsce zerowe $x_0$ & $f(x_0)$ & Liczba iteracji & Kod błędu \\
 		\hline
 		\hline
    		bisekcja & 1.9337539672851562 & -2.7027680138402843e-7 & 16 & 0\\
      	\hline
      	stycznych & 1.933753779789742 & -2.2423316314856834e-8 & 4 & 0\\
      	\hline
      	siecznych & 1.933753644474301 & 1.564525129449379e-7 & 4 & 0\\
		\hline
    \end{tabular}
\end{table}

\noindent \textbf{Wnioski:}\\
Patrząc na wyniki możemy dostrzec iż wszystkie metody wykonały obliczenia poprawnie i nie zwróciły błędu. Najwydajniejsze okazały się metody siecznych i stycznych, co jest zgodne z oczekiwaniami. Metody te mają większą zbieżność od metody bisekcji i wykonały obliczenia w 4 razy mniejszej liczbie iteracji.



\section{Zadanie 5}
Celem zadania 5 jest znalezienie za pomocą metody bisekcji wartości zmiennej $x$, dla której przecinają się wykresy funkcji $y_1=3x$ i $y_2=e^x$. Wymagane dokładności obliczeń wynoszą: $\delta=10^{-4}$, $\epsilon=10^{-4}$.
Aby znaleźć wartość zmiennej $x$ porównujemy podane funkcje, dzięki czemu otrzymujemy równanie:
\begin{equation*}
3x=e^x
\end{equation*}
Zatem będziemy szukać miejsc zerowych funkcji:
\begin{equation*}
f(x)=3x-e^x
\end{equation*}

\begin{figure}[!h]
\minipage{0.49\textwidth}
  \includegraphics[width=\linewidth]{zad5_a_plot.png}
  \caption{Wykres w pyplot dla $y_1$ i $y_2$}\label{fig:figure2}
\endminipage\hfill
\minipage{0.49\textwidth}
  \includegraphics[width=\linewidth]{zad5_b_plot.png}
  \caption{Wykres w pyplot dla $f(x)$}\label{fig:figure3}
\endminipage
\end{figure}

Po wstępnej analizie funkcji można zobaczyć, iż funkcja ma dwa miejsca zerowe. Jako przedziały początkowe dla metody bisekcji wybrałem: $[0.5, 1.0]$ oraz $[1.0, 2.0]$
\\
\\
\noindent \textbf{Wyniki:}\\
Wyniki przedstawia następująca tabela:
\begin{table}[h!]
	\centering
    \label{tab:table2}
    \begin{tabular}{c|c|c|c}
		Miejsce zerowe $x_0$ & $f(x_0)$ & Liczba iteracji & Kod błędu \\
 		\hline
 		\hline
		0.619140625 & 9.066320343276146e-5 & 8 & 0\\
		\hline
		1.5120849609375 & 7.618578602741621e-5 & 13 & 0\\
		\hline
    \end{tabular}
\end{table}
\\
\\
\noindent \textbf{Wnioski:}\\
Algorytm obliczył poszukiwane miejsca zerowe oraz nie zwrócił błędu. Oznacza to że wybraliśmy odpowiednie przedziały. W zadaniu tym główną trudność sprawiało właśnie dobranie odpowiednich przedziałów. W celu znalezienia ich posłużyłem się wykresem funkcji, jednak w przypadku braku dostępu do tego narzędzia musielibyśmy wykonać większą liczbę prób lub użyć innego algorytmu.

\section{Zadanie 6}
Zadanie 6 polega na znalezieniu miejsc zerowych funkcji $f_1(x)=e^{1-x}-1$ oraz $f_2(x)=xe^{-x}$ za pomocą metod bisekcji, Newtona i siecznych. Wymagane dokładności obliczeń wynoszą $\delta=10^{-4}$, $\epsilon=10^{-4}$. Należy dobrać odpowiednio przedział i przybliżenia początkowe. 
\begin{figure}[!h]
	\centering
  \includegraphics[width=0.6\linewidth]{zad6_plot.png}
  \caption{Wykres w pyplot na przedziale $[-1, 10]$}\label{fig:figure4}
\end{figure}

Po wstępnej analizie funkcji można w łatwy sposób dostrzec, że funkcja $f_1$ ma miejsce zerowe w $x_0=1$, a funkcja $f_2$ ma miejsce zerowe w $x_0=0$

\noindent \textbf{Wyniki:}\\
Wyniki przedstawia następujące tabele:
\begin{table}[h!]
	\centering
    \label{tab:table3}
    \begin{tabular}{c|c|c|c|c}
		Przedział & Miejsce zerowe $r$ & $f(r)$ & Liczba iteracji & Kod błędu \\
 		\hline
 		\hline
 		\multicolumn{5}{c}{$f_1$}\\
 		\hline
 		\hline
		$[0.5, 1.5]$ & 1.0 & 0.0 & 1 & 0\\
		\hline
		$[-2.0, 3.0]$ & 0.9999923706054688 & 7.629423635080457e-6 & 17 & 0\\
		\hline
		$[-20.0, 10.0]$ & 0.9999942779541016 & 5.722062269342132e-6 & 20 & 0\\
		\hline
		$[-2000.0, 3000.0]$ & 1.0000094771385193 & -9.477093611320875e-6 & 27 & 0\\
		\hline
		\hline
 		\multicolumn{5}{c}{$f_2$}\\
 		\hline
 		\hline
 		$[-0.5, 0.5]$ & 0.0 & 0.0 & 1 & 0\\
 		\hline
 		$[-2.0, 3.0]$ & 7.62939453125e-6 & 7.62933632381113e-6 & 17 & 0\\
 		\hline
 		$[-20.0, 15.0]$ & -9.5367431640625e-6 & -9.53683411396636e-6 & 19 & 0\\
 		\hline
		$[-2000.0, 3000.0]$ & 500.0 & 3.562288203370643e-215 & 1 & 0\\
 		\hline
    \end{tabular}
	\caption{Metoda bisekcji}
\end{table}

\begin{table}[h!]
	\centering
    \label{tab:table4}
    \begin{tabular}{c|c|c|c|c}
		$x_0$ & Miejsce zerowe $r$ & $f(r)$ & Liczba iteracji & Kod błędu \\
 		\hline
 		\hline
 		\multicolumn{5}{c}{$f_1$}\\
 		\hline
 		\hline
		$1.0$ & 1.0 & 0.0 & 0 & 0\\
		\hline
		$5.0$ & 0.9999996427095682 & 3.572904956339329e-7 & 54 & 0\\
		\hline
		$6.0$ & 0.9999999573590406 & 4.264096031825204e-8 & 147 & 0\\
		\hline
		$7.5$ & 0.9999994332744109 & 5.667257496622113e-7 & 662 & 0\\
		\hline
		$10.0$ & - & - & - & 1 \\
		\hline
		$100.0$ & - & - & - & 2 \\
		\hline
		$1000.0$ & - & - & - & 2 \\
		\hline
		\hline
		\hline
 		\multicolumn{5}{c}{$f_2$}\\
 		\hline
 		\hline
 		$0.0$ & 0.0 & 0.0 & 0 & 0\\
 		\hline
 		$0.5$ & -3.0642493416461764e-7 & -3.0642502806087233e-7 & 5 & 0\\
 		\hline
 		$1.0$ & - & - & - & 2 \\
 		\hline
 		$1.01$ & 102.00999999999992 & 5.0846685549318855e-43 & 1 & 0\\
 		\hline
 		$10.0$ & 14.380524159896261 & 8.173205649825554e-6 & 4 & 0\\
 		\hline
 		$100.0$ & 100.0 & 3.7200759760208363e-42 & 0 & 0\\
 		\hline
 		$1000.0$ & 1000.0 & 0.0 & 0 & 0\\

 		\hline
    \end{tabular}
	\caption{Metoda Newtona dla $10^5$ iteracji}
\end{table}



\begin{table}[h!]
	\centering
    \label{tab:table5}
    \begin{tabular}{c|c|c|c|c}
		$x_0$, $x_1$ & Miejsce zerowe $r$ & $f(r)$ & Liczba iteracji & Kod błędu \\
 		\hline
 		\hline
 		\multicolumn{5}{c}{$f_1$}\\
 		\hline
 		\hline
		$0.5, 1.5$ & 0.9999999624498374 & 3.755016342310569e-8 & 5 & 0\\
		\hline
		$0.0, 3.0$ & 0.999999739048799 & 2.6095123506486573e-7 & 9 & 0\\
		\hline
		$-5.0, 5.0$ & 4.975665372740593 & -0.9812331895845449 & 3 & 0\\
		\hline
		$-10.0, 10.0$ &9.99966600720658 & -0.999876548971044 & 3 & 0\\
		\hline
		$0.0, 100.0$ & - & - & - & 1 \\
		\hline
		\hline
 		\multicolumn{5}{c}{$f_2$}\\
 		\hline
 		\hline
 		$-0,5, 0.5$ & 5.38073548562323e-6 & 5.380706533386756e-6 & 6 & 0\\
 		\hline
 		$0.5, 1.5$ & 5.372227830220525e-6 & 5.372198969466189e-6 & 9 & 0\\
 		\hline
 		$-8.0, 7.0$ & 6.999995985033118 & 0.006383195725973712 & 1 & 0\\
 		\hline
 		$-10.0, 10.0$ & 9.999999958776927 & 0.0004539993144685704 & 1 & 0\\
 		\hline
 		$-1.0, 100.0$ & 100.0 & 3.7200759760208363e-42 & 1 & 0\\
 		\hline
    \end{tabular}
	\caption{Metoda siecznych dla $10^5$ iteracji}
\end{table}
\clearpage
\noindent \textbf{Wnioski:}\\
\indent W metodzie bisekcji w większości przypadków osiągnęliśmy poszukiwany wynik z zadaną dokładnością, nawet mimo wybierania dużych przedziałów. Jedynym wyjątkiem było obliczanie wartości miejsca zerowego dla przedziału $[-2000, 3000]$ dla $f_2$. Dla danego przedziału otrzymaliśmy wartość $500$. Wynika to z faktu, iż wartość funkcji dla środka przedziału w pierwszej iteracji (czyli właśnie dla $x=500$) jest bardzo bliska zeru i spełnia warunek zakończenia obliczeń $|f(r)| < epsilon$. Zatem stosując metodą bisekcji musimy wybierać przedział ostrożnie w przypadku gdy wartości funkcji są bliskie zeru.
\\
\\
\indent W metodzie Newtona dla funkcji $f_1$ dla niewielkich wartości osiągaliśmy poprawne miejsce zerowe w stosunkowo niewielkiej liczbie operacji. Jednak wraz z wzrostem wartości $x_0$ liczba iteracji gwałtownie wzrasta, chociaż wciąż zwraca poprawny wynik. Wynika to z faktu, iż funkcja dla coraz większych wartości wzrasta w coraz wolniejszym tempie, zbliżając się do funkcji stałej. Zatem dla niektórych wartości nie jesteśmy osiągnąć wyniku w zadanej liczbie iteracji - przykładowo dla $x_0 = 10.0$ otrzymujemy error code 1. Dla jeszcze większych wartości $x_0$ pochodna staję się tak bliska zeru, że otrzymujemy error code 2. 

Dla funkcji $f_2$ metoda zwraca poprawne wartości dla wartości mniejszych od $1$. Dla $x_0=1$ pochodna wynosi zero, więc otrzymujemy error code 2. Dla wartości większych od $1$, nawet minimalnie większych, funkcja okazuje się rozbieżna. Algorytm mimo to jednak zwraca pewne wyniki. Wynika to z faktu, iż podobnie jak w metodzie bisekcji spełniony zostaje warunek zakończenia obliczeń $|f(r)| < epsilon$.

Korzystając z metody Newtona musimy zatem odpowiednio dobierać wartość początkową, tak aby metoda była zbieżna. Musimy również zwrócić szczególną uwagę w przypadku gdy funkcja dąży do zera.
\\
\\
\indent Metoda siecznych zachowała się podobnie do metody Newtona. Dla odpowiednio dobranych wartości początkowych zwracała poprawne miejsce zerowe. W funkcji $f_2$ po wybraniu niektórych wartości kończyła pracę po jednej iteracji ze względu na spełnienie warunku $|f(r)| < epsilon$, zwracając wartości bliskie $x_1$. Jedyną różnicą są rozwiązania dla przedziałów $[-5.0, 5.0]$ i $[-10.0, 10.0]$ w $f_1$. Dla podanych przykładów funkcja zwracała wartości zbliżone do $x_1$ ponieważ krańce przedziałów przyjmowały wartości bardzo odległe od siebie, więc następne wartości $x_n$ znajdowały się blisko wartości $x_1$. Algorytm kończył pracę przez spełnienie warunku $|x_{n+1}-x_n| < delta$. Korzystając z metody siecznych musimy zatem uważać, gdy wybieramy wartości np. dla funkcji, która dąży do zera lub gdy wartości na krańcach przedziału znacznie się różnią.


\end{document}